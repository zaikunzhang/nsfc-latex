% 定义字体。*.ttf 为字体文件名。字体文件由使用者提供并置于当前目录下。
%\setCJKmainfont[BoldFont={[simhei.ttf]},ItalicFont={[simkai.ttf]}]{[simsun.ttf]}
%\setCJKmainfont[BoldFont={[simhei.ttf]},ItalicFont={[simkai.ttf]}]{[SimSun]}
%\setCJKmainfont[BoldFont={[simhei]},ItalicFont={[simkai]}]{[simsun]}
\setCJKsansfont{[simhei]}
\setCJKmonofont{[simfang]}

\setCJKfamilyfont{zhsong}{[simsun]}
%\setCJKfamilyfont{zhhei}{[noto-l.otf]}
%\setCJKfamilyfont{zhhei}{[Noto Sans CJK SC]}
\setCJKfamilyfont{zhhei}{[simihei]}
\setCJKfamilyfont{zhkai}{[simkai]}
\setCJKfamilyfont{zhfs}{[simfang]}
\setCJKfamilyfont{zhli}{[simli.ttf]}
\setCJKfamilyfont{zhyou}{[simyou.ttf]}
\setCJKfamilyfont{zhxingkai}{[stxingkai.ttf]}
\setCJKfamilyfont{zhxinwei}{[stxinwei.ttf]}

% 定义字体调用命令。
\renewcommand*{\songti}{\CJKfamily{zhsong}} % 宋体
\renewcommand*{\heiti}{\CJKfamily{zhhei}}   % 黑体
\newcommand*{\kaiti}{\CJKfamily{zhkai}}  % 楷书
\renewcommand*{\fangsong}{\CJKfamily{zhfs}} % 仿宋
\newcommand*{\liti}{\CJKfamily{zhli}}    % 隶书
\newcommand*{\youyuan}{\CJKfamily{zhyou}} % 幼圆
\newcommand*{\xinwei}{\CJKfamily{zhxinwei}}    % 新魏
\newcommand*{\xingkai}{\CJKfamily{zhxingkai}}    % 行楷
